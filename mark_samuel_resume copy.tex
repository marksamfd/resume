%-------------------------
% Resume in LateX
% Author : Sourabh Bajaj
% License : MIT
%------------------------

\documentclass[letterpaper,11pt]{article}

\usepackage{latexsym}
\usepackage[empty]{fullpage}
\usepackage{titlesec}
\usepackage{marvosym}
\usepackage[usenames,dvipsnames]{color}
\usepackage{verbatim}
\usepackage{enumitem}
\usepackage[hidelinks]{hyperref}
\usepackage{fancyhdr}
\usepackage[english]{babel}
\usepackage{tabularx}
\input{glyphtounicode}

\pagestyle{fancy}
\fancyhf{} % Clear all header and footer fields
\fancyfoot{}
\renewcommand{\headrulewidth}{0pt}
\renewcommand{\footrulewidth}{0pt}

% Adjust margins
\addtolength{\oddsidemargin}{-0.5in}
\addtolength{\evensidemargin}{-0.5in}
\addtolength{\textwidth}{1in}
\addtolength{\topmargin}{-.5in}
\addtolength{\textheight}{1.0in}

\urlstyle{same}

\raggedbottom
\raggedright
\setlength{\tabcolsep}{0in}

% Sections formatting
\titleformat{\section}{
  \vspace{-4pt}\scshape\raggedright\large
}{}{0em}{}[\color{black}\titlerule \vspace{-5pt}]

% Ensure that generate PDF is machine readable/ATS parsable
\pdfgentounicode=1

%-------------------------
% Custom commands
\newcommand{\resumeItem}[2]{
  \item\small{
    \textbf{#1}{: #2 \vspace{-2pt}}
  }
}

% Just in case someone needs a heading that does not need to be in a list
\newcommand{\resumeHeading}[4]{
    \begin{tabular*}{0.99\textwidth}[t]{l@{\extracolsep{\fill}}r}
      \textbf{#1} & #2 \\
      \textit{\small#3} & \textit{\small #4} \\
    \end{tabular*}\vspace{-5pt}
}

\newcommand{\resumeSubheading}[4]{
  \vspace{-1pt}\item
    \begin{tabular*}{0.97\textwidth}[t]{l@{\extracolsep{\fill}}r}
      \textbf{#1} & #2 \\
      \textit{\small#3} & \textit{\small #4} \\
    \end{tabular*}\vspace{-5pt}
}

\newcommand{\resumeSubSubheading}[2]{
    \begin{tabular*}{0.97\textwidth}{l@{\extracolsep{\fill}}r}
      \textit{\small#1} & \textit{\small #2} \\
    \end{tabular*}\vspace{-5pt}
}

\newcommand{\resumeSubItem}[2]{\resumeItem{#1}{#2}\vspace{-4pt}}

\renewcommand{\labelitemii}{$\circ$}

\newcommand{\resumeSubHeadingListStart}{\begin{itemize}[leftmargin=*]}
\newcommand{\resumeSubHeadingListEnd}{\end{itemize}}
\newcommand{\resumeItemListStart}{\begin{itemize}}
\newcommand{\resumeItemListEnd}{\end{itemize}\vspace{-5pt}}

%-------------------------------------------
%%%%%%  CV STARTS HERE  %%%%%%%%%%%%%%%%%%%%%%%%%%%%


\begin{document}

%----------HEADING-----------------
\begin{tabular*}{\textwidth}{l@{\extracolsep{\fill}}r}
  \textbf{\href{https://}{\Large Mark Samuel}} & Email: \href{mailto:s-mark.kirelos@zewailcity.edu.eg}{s-mark.kirelos@zewailcity.edu.eg}\\
  \href{https://marksamfd.github.io}{Website} -- \href{https://www.linkedin.com/in/marksamfd}{LinkedIn} -- \href{https://github.com/marksamfd/}{Github} -- \href{https://open.spotify.com/show/3Mw8eB3BbXNyIzRruPxyHc}{Podcast} & Mobile: \href{tel:+201200129803}{+20 120 012 9803}\\
\end{tabular*}


%-----------EDUCATION-----------------
\section{Education}
  \resumeSubHeadingListStart
  % \resumeSubheading
  %   {Birla Institute of Technology and Science}{Pilani, India}
  %   {Master Of Engineering in Electrical and Electronics; GPA: 3.66 (9.15/10.0)}{Aug 2008 -- July 2012}
    \resumeSubheading
      {University of Science and Technology at Zewail City}{October, Egypt}
      {Bachelor of Computer Science Majoring in Data Science and AI}{Oct 2022 -- Now}
    \resumeSubSubheading{Relevant Coursework}{
      \begin{itemize}
        \item Machine learning
        \item Deep learning
        \item Natural Language processing (NLP)
        \item Computer Vision
      \end{itemize}
    }
  \resumeSubHeadingListEnd


%-----------EXPERIENCE-----------------
\section{Experience}
  \resumeSubHeadingListStart

    \resumeSubheading
      {ISchool}{Remote}
      {Coding Instructor}{June 2024 -- Aug 2024}
      \resumeItemListStart
        \resumeItem{Programming}
          {Taught 6th grades basic programming and basic concepts.}
        \resumeItem{Python}
          {Taught 6th grades python.}
        \resumeItem{AI and ML}
          {Taught 6th grades basics of AI and ML.}
      \resumeItemListEnd
      
    \resumeSubheading
      {Steigenberger AL DAU Beach Hotel}{Hurghada, EG}
      {IT Trainee}{June 2023 -- Aug 2023}
      \resumeItemListStart
        \resumeItem{Hotel Infrastructure}
          {Understood hotel’s infrastructure.}
        \resumeItem{PC Maintainance}
          {Maintained the PCs and other devices.}
        \resumeItem{Network devices}
          {Maintained network devices and cables.}
      \resumeItemListEnd
      
% --------Multiple Positions Heading------------
  %  \resumeSubSubheading
  %   {Software Engineer I}{Oct 2014 -- Sep 2016}
  %   \resumeItemListStart
  %      \resumeItem{Apache Beam}
  %        {Apache Beam is a unified model for defining both batch and streaming data-parallel processing pipelines}
  %   \resumeItemListEnd

%-------------------------------------------

    

  \resumeSubHeadingListEnd


%-----------PROJECTS-----------------
\section{Projects}
\resumeSubHeadingListStart
    \resumeSubItem{Explainable AI for Skin Cancer Detection (\href{https://github.com/haneenalaa465/Explainable-AI-for-Skin-Cancer-Detection/}{Github})}
      {Developed an XAI-based diagnostic system achieving 92\% accuracy on the HAM10000 dataset using PyTorch. Implemented CNN architectures (ResNet50, EfficientNet-B0) and integrated explainability methods (LIME, SHAP, Grad-CAM) resulting in 85\% interpretability score by dermatologists.}
    \resumeSubItem{Deep Fake Image Detector (\href{https://github.com/AhmedFoda54/Deep-Fake-Classification}{Github})}
      {Engineered a CNN-based detection system achieving 99\% accuracy on a custom dataset of 50,000 images. Implemented ensemble learning with Inception-v3 and ResNet-50, reducing false positives by 30\% compared to single-model approaches.}
    \resumeSubItem{Activity Detection Using MHealth (\href{https://github.com/marksamfd/mhealth-classification}{Github})}
      {A machine learning project for classifying user activities based on accelerometer and gyroscope data. Includes EDA for the sensor data using Pandas and Matplotlib. Utilizes algorithms such as KNN, SVM, and Logistic Regression using Scikit-Learn and Neural Networks using TensorFlow.}
    \resumeSubItem{Adagrad TensorFlow Implementation (\href{https://github.com/marksamfd/AdaGrad-Math-303}{Github})}
      {An implementation of the Adagrad optimization algorithm from scratch as a TensorFlow optimizer, showcasing a deep understanding of optimization techniques. as a project for Linear and non linear programming course}
    \resumeSubItem{YouAttend, Face Attendance Application (\href{https://github.com/marksamfd/YouAttend}{Github})}
      {A cross-platform, local-first React Native app for face recognition-based attendance tracking. Powered by Facenet with TFLite and ONNX frameworks (Java implementation), and stores attendance data in a vector database that is based on SQLite.}
    \resumeSubItem{Company Dashboard (\href{https://github.com/marksamfd/Data-Analysis-Dashboard}{Github})}
      {A data visualization dashboard for analyzing company performance trends over time. Built with Python, Flask, and AmCharts to deliver interactive and insightful visualizations.}
   
    \resumeSubItem{Video Slideshow (\href{https://github.com/marksamfd/VideoSlideshow}{Github})}
      {A cross-platform Electron app for creating dynamic video slideshows. It allows users to bind text slides to video backgrounds using Node.js, HTML, CSS, and FFmpeg.}
    \resumeSubItem{Schedule to Ical (\href{https://github.com/marksamfd/SS-to-ical}{Github})}
      {A web scraper that converts schedules from Ellucian\texttrademark Self-Service into ICS files, enabling seamless integration with calendar providers. Uses puppeteer to scrape your self-service dashboard to get the data}
    \resumeSubItem{Cakery (\href{https://github.com/marksamfd/CakeryMarkDEV}{Github})}
      {A full-stack software engineering project featuring Python and flask for the server, PostgreSQL, Firebase push notifications, Next.js, Docker, and CI/CD pipelines to deliver robust and scalable solutions.}
    \resumeSubItem{Smart watch Reverse Engineering (\href{https://}{Github})}
      {Reverse engineered a smart watch to get heartrate data on my PC. Utilized Bluetooth Low Energy (BLE) and  Generic ATTribute Profile (GATT) to read the data. Utilized electron to create cross-platform app. HTML, CSS, Bootstrap, and JS}

\resumeSubHeadingListEnd


%-----------SKILLS------------
\section{Skills}
 \resumeSubHeadingListStart
   \item{
     \textbf{Machine Learning/AI}{: PyTorch, TensorFlow, Scikit-Learn, Deep Learning, CNNs, Transfer Learning, XAI}
     \hfill
     \textbf{Data Science}{: Pandas, NumPy, Matplotlib, Seaborn, SQL, Data Preprocessing}
   }
   \item{
     \textbf{Programming}{: Python, C++, JavaScript, Java, Kotlin}
     \hfill
     \textbf{Tools \& Platforms}{: Git, Docker, Linux, AWS, Flask, React}
   }
 \resumeSubHeadingListEnd


%-------------------------------------------
\end{document}
